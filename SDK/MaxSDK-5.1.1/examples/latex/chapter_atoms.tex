When a Max object receives a message, it uses its class to look up the message selector (\char`\"{}int\char`\"{}, \char`\"{}bang\char`\"{}, \char`\"{}set\char`\"{} etc.

) and invoke the associated C function (method). This association is what you are creating when you use \hyperlink{group__class_ga1fabf54e0cec8d4e5f732fa347b3f874}{class\_\-addmethod()} in the initialization routine. If the lookup fails, you'll see an \char`\"{}object doesn't understand message\char`\"{} error in the Max window.

Message selectors are not character strings, but a special data structure called a symbol (\hyperlink{structt__symbol}{t\_\-symbol}). A symbol holds a string and a value, but what is more important is that every symbol in Max is unique. This permits you to compare two symbols for equivalence by comparing pointers, rather than having to compare each character in two strings.

The \char`\"{}data\char`\"{} or argument part of a message, if it exists, is transmitted in the form of an array of atoms (\hyperlink{structt__atom}{t\_\-atom}). The atom is a structure that can hold integers, floats, symbols, or even pointers to other objects, identified by a tag. You'll use symbols and atoms both in sending messages and receiving them.

To illustrate the use of symbols and atoms, here is how you would send a message out an outlet. Let's say we want to send the message \char`\"{}green 43 crazy 8.34.\char`\"{} This message consists of a selector \char`\"{}green\char`\"{} plus an array of three atoms.

First, we'll need to create a generic outlet with outlet\_\-new in our new instance routine. 
\begin{DoxyCode}
        x->m_outlet = outlet_new((t_object *)x, NULL);
\end{DoxyCode}


The second argument being NULL indicates that the outlet can be used to send any message. If the second argument had been a character string such as \char`\"{}int\char`\"{} or \char`\"{}set\char`\"{} only that specific message could be sent out the outlet. You'd be correct if you wondered whether \hyperlink{group__inout_ga9b8d897c728eeafa5638d4fc16ff704e}{intout()} is actually just outlet\_\-new(x, \char`\"{}int\char`\"{}).

Now that we have our generic outlet, we'll call \hyperlink{group__inout_ga12798ee897e01dac21ee547c4091d8a8}{outlet\_\-anything()} on it in a method. The first step, however, is to assemble our message, with a selector \char`\"{}green\char`\"{} plus an array of atoms. Assigning ints and floats to an atom is relatively simple, but to assign a symbol, we need to transform a character string into a symbol using \hyperlink{group__symbol_ga8268797d125a15bae1010af70b559e05}{gensym()}. The \hyperlink{group__symbol_ga8268797d125a15bae1010af70b559e05}{gensym()} function returns a pointer to a symbol that is guaranteed to be unique for the string you supply. This means the string is compared with other symbols to ensure its uniqueness. If it already exists, \hyperlink{group__symbol_ga8268797d125a15bae1010af70b559e05}{gensym()} will supply a pointer to the symbol. Otherwise it will create a new one and store it in a table so it can be found the next time someone asks for it.


\begin{DoxyCode}
    void myobject_bang(t_object *x)
    {
        t_atom argv[3];

        atom_setlong(argv, 43);
        atom_setsym(argv + 1, gensym("crazy"));
        atom_setfloat(argv + 2, 8.34);

        outlet_anything(x->m_outlet, gensym("green"), 3, argv);
    }
\end{DoxyCode}


In the call to \hyperlink{group__inout_ga12798ee897e01dac21ee547c4091d8a8}{outlet\_\-anything()} above, gensym(\char`\"{}green\char`\"{}) represents the message selector. The \hyperlink{group__inout_ga12798ee897e01dac21ee547c4091d8a8}{outlet\_\-anything()} function will try to find a message \char`\"{}green\char`\"{} in each of the objects connected to the inlet. If \hyperlink{group__inout_ga12798ee897e01dac21ee547c4091d8a8}{outlet\_\-anything()} finds such a message, it will execute it, passing it the array of atoms it received.

If it cannot find a match for the symbol green, it does one more thing, which allows objects to handle messages generically. Your object can define a special method bound to the symbol \char`\"{}anything\char`\"{} that will be invoked if no other match is found for a selector. We'll discuss the anything method in a moment, but first, we need to return to \hyperlink{group__class_ga1fabf54e0cec8d4e5f732fa347b3f874}{class\_\-addmethod()} and explain the final arguments it accepts.

To access atoms, you can use the functions \hyperlink{group__atom_ga98af493b18dfac0f8d441e16e520d5f6}{atom\_\-setlong()}, \hyperlink{group__atom_ga62c0a631f50db54ec654a9e40b992fe2}{atom\_\-getlong()} etc. or you can access the \hyperlink{structt__atom}{t\_\-atom} structure directly. We recommend using the accessor functions, as they lead to both cleaner code and will permit your source to work without modifications when changes to the \hyperlink{structt__atom}{t\_\-atom} structure occur over time.\hypertarget{chapter_atoms_chapter_atoms_types}{}\section{Argument Type Specifiers}\label{chapter_atoms_chapter_atoms_types}
In the simp example, you saw the int method defined as follows: 
\begin{DoxyCode}
        class_addmethod(c, (method)simp_int, A_LONG, 0);
\end{DoxyCode}


The \hyperlink{group__atom_gga8aa6700e9f00b132eb376db6e39ade47a002f28879581a6f66ea492b994b96f1e}{A\_\-LONG}, 0 arguments to \hyperlink{group__class_ga1fabf54e0cec8d4e5f732fa347b3f874}{class\_\-addmethod()} specify the type of arguments expected by the C function you have written. \hyperlink{group__atom_gga8aa6700e9f00b132eb376db6e39ade47a002f28879581a6f66ea492b994b96f1e}{A\_\-LONG} means that the C function accepts a long integer argument. The 0 terminates the argument specifier list, so for the int message, there is a single long integer argument.

The other options are \hyperlink{group__atom_gga8aa6700e9f00b132eb376db6e39ade47a0b3aa0ab8104573dfc9cb70b5b08031f}{A\_\-FLOAT} for doubles, \hyperlink{group__atom_gga8aa6700e9f00b132eb376db6e39ade47a2d661c2a5d949566e2f1944c99bceeea}{A\_\-SYM} for symbols, and \hyperlink{group__atom_gga8aa6700e9f00b132eb376db6e39ade47a81c1a8550f038db16a619167a70a79b6}{A\_\-GIMME}, which passes the raw list of atoms that were originally used to send the Max message in the first place. These argument type specifiers define what are known as \char`\"{}typed\char`\"{} methods in Max. Typed methods are those where Max checks the type of each atom in a message to ensure it is consistent with what the receiving object has said it expects for a given selector.

If the atoms cannot be coerced into the format of the argument type specifier, a bad arguments error is printed in the Max window.

There is a limit to the number of specifiers you can use, and in general, multiple \hyperlink{group__atom_gga8aa6700e9f00b132eb376db6e39ade47a0b3aa0ab8104573dfc9cb70b5b08031f}{A\_\-FLOAT} specifiers should be avoided due to the historically unpredictable nature of compiler implementations when passing floating-\/point values on the stack. Use \hyperlink{group__atom_gga8aa6700e9f00b132eb376db6e39ade47a81c1a8550f038db16a619167a70a79b6}{A\_\-GIMME} for more than four arguments or with multiple floating-\/point arguments.

You can also specify that missing arguments to a message be filled in with default values before your C function receives them. \hyperlink{group__atom_gga8aa6700e9f00b132eb376db6e39ade47a7bd979db3dcf86909e24a1d1452e2205}{A\_\-DEFLONG} will put a 0 in place of a missing long argument, \hyperlink{group__atom_gga8aa6700e9f00b132eb376db6e39ade47a42b644240dcbb90fe67282a4d0688776}{A\_\-DEFFLOAT} will put 0.0 in place of a missing float argument, and \hyperlink{group__atom_gga8aa6700e9f00b132eb376db6e39ade47aa010616276cb89bcd04bcba611e18d51}{A\_\-DEFSYM} will put the empty symbol (equal to gensym(\char`\"{}\char`\"{})) in place of a missing symbol argument.\hypertarget{chapter_atoms_chapter_atoms_gimme_funcs}{}\section{Writing A\_\-GIMME Functions}\label{chapter_atoms_chapter_atoms_gimme_funcs}
A method that uses \hyperlink{group__atom_gga8aa6700e9f00b132eb376db6e39ade47a81c1a8550f038db16a619167a70a79b6}{A\_\-GIMME} is declared as follows: 
\begin{DoxyCode}
    void myobject_message(t_myobject *x, t_symbol *s, long argc, t_atom *argv);
\end{DoxyCode}


The symbol argument s is the message selector. Ordinarily this might seem redundant, but it is useful for the \char`\"{}anything\char`\"{} method as we'll discuss below.

argc is the number of atoms in the argv array. It could be 0 if the message was sent without arguments. argv is the array of atoms holding the arguments.

For typed messages, the atoms will be of type \hyperlink{group__atom_gga8aa6700e9f00b132eb376db6e39ade47a2d661c2a5d949566e2f1944c99bceeea}{A\_\-SYM}, \hyperlink{group__atom_gga8aa6700e9f00b132eb376db6e39ade47a0b3aa0ab8104573dfc9cb70b5b08031f}{A\_\-FLOAT}, or \hyperlink{group__atom_gga8aa6700e9f00b132eb376db6e39ade47a002f28879581a6f66ea492b994b96f1e}{A\_\-LONG}. Here is an example of a method that merely prints all of the arguments.


\begin{DoxyCode}
    void myobject_printargs(t_myobject *x, t_symbol *s, long argc, t_atom *argv)
    {
        long i;
        t_atom *ap;

        post("message selector is %s",s->s_name);
        post("there are %ld arguments",argc);
        for (i = 0, ap = argv; i < argc; i++, ap++) {       // increment ap each 
      time to get to the next atom
            switch (atom_gettype(ap)) {
                case A_LONG:
                    post("%ld: %ld",i+1,atom_getlong(ap));
                    break;
                case A_FLOAT:
                    post("%ld: %.2f",i+1,atom_getfloat(ap));
                    break;
                case A_SYM:
                    post("%ld: %s",i+1, atom_getsym(ap)->s_name);
                    break;
                default:
                    post("%ld: unknown atom type (%ld)", i+1, atom_gettype(ap));
                    break;
            }
        }
    }
\end{DoxyCode}


You can interpret the arguments in whatever manner you wish. You cannot, however, modify the arguments as they may be about to be passed to another object. 